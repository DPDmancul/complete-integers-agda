% Options for packages loaded elsewhere
\PassOptionsToPackage{unicode}{hyperref}
\PassOptionsToPackage{hyphens}{url}
%
\documentclass[
]{article}
\usepackage{amsmath,amssymb}
\usepackage{lmodern}
\usepackage{iftex}
\ifPDFTeX
  \usepackage[T1]{fontenc}
  \usepackage[utf8]{inputenc}
  \usepackage{textcomp} % provide euro and other symbols
\else % if luatex or xetex
  \usepackage{unicode-math}
  \defaultfontfeatures{Scale=MatchLowercase}
  \defaultfontfeatures[\rmfamily]{Ligatures=TeX,Scale=1}
\fi
% Use upquote if available, for straight quotes in verbatim environments
\IfFileExists{upquote.sty}{\usepackage{upquote}}{}
\IfFileExists{microtype.sty}{% use microtype if available
  \usepackage[]{microtype}
  \UseMicrotypeSet[protrusion]{basicmath} % disable protrusion for tt fonts
}{}
\makeatletter
\@ifundefined{KOMAClassName}{% if non-KOMA class
  \IfFileExists{parskip.sty}{%
    \usepackage{parskip}
  }{% else
    \setlength{\parindent}{0pt}
    \setlength{\parskip}{6pt plus 2pt minus 1pt}}
}{% if KOMA class
  \KOMAoptions{parskip=half}}
\makeatother
\usepackage{xcolor}
\IfFileExists{xurl.sty}{\usepackage{xurl}}{} % add URL line breaks if available
\IfFileExists{bookmark.sty}{\usepackage{bookmark}}{\usepackage{hyperref}}
\hypersetup{
  hidelinks,
  pdfcreator={LaTeX via pandoc}}
\urlstyle{same} % disable monospaced font for URLs
\usepackage[margin=1in]{geometry}
\usepackage{color}
\usepackage{fancyvrb}
\newcommand{\VerbBar}{|}
\newcommand{\VERB}{\Verb[commandchars=\\\{\}]}
\DefineVerbatimEnvironment{Highlighting}{Verbatim}{commandchars=\\\{\}}
% Add ',fontsize=\small' for more characters per line
\usepackage{framed}
\definecolor{shadecolor}{RGB}{248,248,248}
\newenvironment{Shaded}{\begin{snugshade}}{\end{snugshade}}
\newcommand{\AlertTok}[1]{\textcolor[rgb]{0.94,0.16,0.16}{#1}}
\newcommand{\AnnotationTok}[1]{\textcolor[rgb]{0.56,0.35,0.01}{\textbf{\textit{#1}}}}
\newcommand{\AttributeTok}[1]{\textcolor[rgb]{0.77,0.63,0.00}{#1}}
\newcommand{\BaseNTok}[1]{\textcolor[rgb]{0.00,0.00,0.81}{#1}}
\newcommand{\BuiltInTok}[1]{#1}
\newcommand{\CharTok}[1]{\textcolor[rgb]{0.31,0.60,0.02}{#1}}
\newcommand{\CommentTok}[1]{\textcolor[rgb]{0.56,0.35,0.01}{\textit{#1}}}
\newcommand{\CommentVarTok}[1]{\textcolor[rgb]{0.56,0.35,0.01}{\textbf{\textit{#1}}}}
\newcommand{\ConstantTok}[1]{\textcolor[rgb]{0.00,0.00,0.00}{#1}}
\newcommand{\ControlFlowTok}[1]{\textcolor[rgb]{0.13,0.29,0.53}{\textbf{#1}}}
\newcommand{\DataTypeTok}[1]{\textcolor[rgb]{0.13,0.29,0.53}{#1}}
\newcommand{\DecValTok}[1]{\textcolor[rgb]{0.00,0.00,0.81}{#1}}
\newcommand{\DocumentationTok}[1]{\textcolor[rgb]{0.56,0.35,0.01}{\textbf{\textit{#1}}}}
\newcommand{\ErrorTok}[1]{\textcolor[rgb]{0.64,0.00,0.00}{\textbf{#1}}}
\newcommand{\ExtensionTok}[1]{#1}
\newcommand{\FloatTok}[1]{\textcolor[rgb]{0.00,0.00,0.81}{#1}}
\newcommand{\FunctionTok}[1]{\textcolor[rgb]{0.00,0.00,0.00}{#1}}
\newcommand{\ImportTok}[1]{#1}
\newcommand{\InformationTok}[1]{\textcolor[rgb]{0.56,0.35,0.01}{\textbf{\textit{#1}}}}
\newcommand{\KeywordTok}[1]{\textcolor[rgb]{0.13,0.29,0.53}{\textbf{#1}}}
\newcommand{\NormalTok}[1]{#1}
\newcommand{\OperatorTok}[1]{\textcolor[rgb]{0.81,0.36,0.00}{\textbf{#1}}}
\newcommand{\OtherTok}[1]{\textcolor[rgb]{0.56,0.35,0.01}{#1}}
\newcommand{\PreprocessorTok}[1]{\textcolor[rgb]{0.56,0.35,0.01}{\textit{#1}}}
\newcommand{\RegionMarkerTok}[1]{#1}
\newcommand{\SpecialCharTok}[1]{\textcolor[rgb]{0.00,0.00,0.00}{#1}}
\newcommand{\SpecialStringTok}[1]{\textcolor[rgb]{0.31,0.60,0.02}{#1}}
\newcommand{\StringTok}[1]{\textcolor[rgb]{0.31,0.60,0.02}{#1}}
\newcommand{\VariableTok}[1]{\textcolor[rgb]{0.00,0.00,0.00}{#1}}
\newcommand{\VerbatimStringTok}[1]{\textcolor[rgb]{0.31,0.60,0.02}{#1}}
\newcommand{\WarningTok}[1]{\textcolor[rgb]{0.56,0.35,0.01}{\textbf{\textit{#1}}}}
\usepackage{longtable,booktabs,array}
\usepackage{calc} % for calculating minipage widths
% Correct order of tables after \paragraph or \subparagraph
\usepackage{etoolbox}
\makeatletter
\patchcmd\longtable{\par}{\if@noskipsec\mbox{}\fi\par}{}{}
\makeatother
% Allow footnotes in longtable head/foot
\IfFileExists{footnotehyper.sty}{\usepackage{footnotehyper}}{\usepackage{footnote}}
\makesavenoteenv{longtable}
\usepackage{graphicx}
\makeatletter
\def\maxwidth{\ifdim\Gin@nat@width>\linewidth\linewidth\else\Gin@nat@width\fi}
\def\maxheight{\ifdim\Gin@nat@height>\textheight\textheight\else\Gin@nat@height\fi}
\makeatother
% Scale images if necessary, so that they will not overflow the page
% margins by default, and it is still possible to overwrite the defaults
% using explicit options in \includegraphics[width, height, ...]{}
\setkeys{Gin}{width=\maxwidth,height=\maxheight,keepaspectratio}
% Set default figure placement to htbp
\makeatletter
\def\fps@figure{htbp}
\makeatother
\setlength{\emergencystretch}{3em} % prevent overfull lines
\providecommand{\tightlist}{%
  \setlength{\itemsep}{0pt}\setlength{\parskip}{0pt}}
\setcounter{secnumdepth}{5}
\ifLuaTeX
  \usepackage{selnolig}  % disable illegal ligatures
\fi

\author{}
\date{\vspace{-2.5em}}

\usepackage{amsthm}
\newtheorem{theorem}{Theorem}[section]
\newtheorem{lemma}{Lemma}[section]
\newtheorem{corollary}{Corollary}[section]
\newtheorem{proposition}{Proposition}[section]
\newtheorem{conjecture}{Conjecture}[section]
\theoremstyle{definition}
\newtheorem{definition}{Definition}[section]
\theoremstyle{definition}
\newtheorem{example}{Example}[section]
\theoremstyle{definition}
\newtheorem{exercise}{Exercise}[section]
\theoremstyle{definition}
\newtheorem{hypothesis}{Hypothesis}[section]
\theoremstyle{remark}
\newtheorem*{remark}{Remark}
\newtheorem*{solution}{Solution}
\begin{document}

{
\setcounter{tocdepth}{2}
\tableofcontents
}
\hypertarget{complete-integers}{%
\section{Complete integers}\label{complete-integers}}

\begin{Shaded}
\begin{Highlighting}[]
\CommentTok{{-}{-} (c) Davide Peressoni 2022}

\KeywordTok{open} \KeywordTok{import}\NormalTok{ Int}
\KeywordTok{open} \KeywordTok{import}\NormalTok{ F2}
\end{Highlighting}
\end{Shaded}

\hypertarget{complete-integer-numbers}{%
\subsection{Complete integer numbers}\label{complete-integer-numbers}}

\begin{definition}[Complete integers prime]

Let's define the set of the complete integer numbers as

\[\mathbb{Z}_C' \coloneqq \mathbb{Z}\times\mathbb{F}_2\]

We will call the first component \emph{value}, and the second \emph{parity}.

\begin{Shaded}
\begin{Highlighting}[]
\KeywordTok{record}\NormalTok{ ℤC\textquotesingle{} }\OtherTok{:} \DataTypeTok{Set} \KeywordTok{where}
  \KeywordTok{constructor}\NormalTok{ [}\OtherTok{\_}\NormalTok{,}\OtherTok{\_}\NormalTok{]}
  \KeywordTok{field}
\NormalTok{    val }\OtherTok{:}\NormalTok{ ℤ}
\NormalTok{    par }\OtherTok{:}\NormalTok{ 𝔽₂}
\end{Highlighting}
\end{Shaded}

\end{definition}

\begin{definition}[Ring $\mathbb{Z}_C'$]

Let's define \(\mathbb{Z}_C'\) as a commutative ring with unit:

Given \([a,b], [c,d] \in \mathbb{Z}_C'\)

\[[a,b] + [c,d] \coloneqq [a+c, b\oplus d]\]

\begin{Shaded}
\begin{Highlighting}[]
\KeywordTok{instance}
  \KeywordTok{open} \KeywordTok{import}\NormalTok{ Ops}
  \KeywordTok{open}\NormalTok{ Sum ⦃ }\OtherTok{...}\NormalTok{ ⦄ }\KeywordTok{public}
\NormalTok{  SumℤC\textquotesingle{} }\OtherTok{:}\NormalTok{ Sum ℤC\textquotesingle{}}
  \OtherTok{\_}\NormalTok{+}\OtherTok{\_}\NormalTok{ ⦃ SumℤC\textquotesingle{} ⦄ [ a , b ] [ c , d ] }\OtherTok{=}\NormalTok{ [ a Int}\OtherTok{.}\NormalTok{+ c , b ⊕ d ]}
\end{Highlighting}
\end{Shaded}

\[[a,b] \cdot [c,d] \coloneqq [a\cdot c, b\cdot d]\]

\begin{Shaded}
\begin{Highlighting}[]
\KeywordTok{instance}
  \KeywordTok{open} \KeywordTok{import}\NormalTok{ Ops}
  \KeywordTok{open}\NormalTok{ Mul ⦃ }\OtherTok{...}\NormalTok{ ⦄ }\KeywordTok{public}
\NormalTok{  MulℤC\textquotesingle{} }\OtherTok{:}\NormalTok{ Mul ℤC\textquotesingle{}}
  \OtherTok{\_}\NormalTok{·}\OtherTok{\_}\NormalTok{ ⦃ MulℤC\textquotesingle{} ⦄ [ a , b ] [ c , d ] }\OtherTok{=}\NormalTok{ [ a Int}\OtherTok{.}\NormalTok{· c , b F2}\OtherTok{.}\NormalTok{· d ]}
\end{Highlighting}
\end{Shaded}

\end{definition}

\end{document}
